\documentclass[obeyspaces,spaces,hyphens]{beamer}
\usepackage[utf8]{inputenc}

%\mode<presentation>

\begin{document}
\title{Bases de datos no tradicionales - clase 1}
\author{Felipe Gorostiaga - Guido Martínez}
\institute{Bases de datos avanzadas, LCC}

\begin{frame}
  \titlepage
\end{frame}

\section{Bases de datos temporales}

\begin{frame}
	\begin{enumerate}
	\item {\bf Bases de datos temporales}
	\item Bases de datos espaciales
	\item Bases de datos espacio-temporales
	\end{enumerate}
\end{frame}

\subsection{Motivación}

\begin{frame}
	\frametitle{Motivación}
	\begin{itemize}
	\item	Muchas aplicaciones de las bases de datos deben tener
		en cuenta la temporalidad de los datos.
		\pause

	\item	Si bien se puede manejar desde un nivel mas alto, esto
		complica las consultas SQL y se vuelve inmanejable rápidamente.
		\pause

	\item	Por eso, queremos tener motores que sean ``conscientes'' de
		la temporalidad, con un lenguaje especializado.
	\end{itemize}
\end{frame}

\begin{frame}
	\frametitle{Definición de DB Temporal}
	\begin{itemize}
	\item	Una DB Temporal es una DB con soporte interno para manejar
		datos temporales.
	\end{itemize}
\end{frame}

\begin{frame}
\begin{center}
	¿Dudas?
	\pause

	¿Quejas?
\end{center}
\end{frame}

\end{document}

% Motivación y definición
% Tiempo válido y tiempo de transacción
% Gráficos de http://www.timeconsult.com/TemporalData/TemporalDB.html
