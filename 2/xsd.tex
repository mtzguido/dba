\section{XSD}

\begin{frame}
\frametitle{Motivación}
\begin{itemize}
\item	Si bien XML no impone ninguna restricción a la presencia o cantidad
	de los elementos, a veces queremos rechazar ciertos documentos
	como inválidos.
	\pause

\item	Se necesita un sistema de tipos para XML. La solución
	estándar es XSD (XML Schema Document)
	\pause

\item	En el documento XSD podemos especificar la estructura que
	debe tener el documento para ser válido.
	\pause

\item	Un documento XSD: ¡también es XML!
\end{itemize}
\end{frame}

\begin{frame}
\frametitle{Características}
\begin{itemize}
\item	Permite definir tipos para elementos XML, tanto {\it simples}
	(números y strings, con posibles restricciones) cómo
	{\it complejos} (secuencias, ``choice'', etc.)
	\pause

\item	\texttt{xsd:all}: representa que todos los subelementos
	deben aparecer en el documento, en cualquier orden. También
	se permiten elementos opcionales.
	\pause

\item	\texttt{xsd:sequence}: similar a \texttt{all}, pero enfuerza
	el orden.
	\pause

\item	\texttt{xsd:choice}: uno y sólo uno de los subelementos
	matchea.
\end{itemize}
\end{frame}

\begin{frame}
\frametitle{Ejemplos - 1}
\footnotesize
\texttt{<xsd:complexType name='USAddress'>				\\
	~~<xsd:sequence>						\\
	~~~~<xsd:element name='name'   type='xsd:string'/>		\\
	~~~~<xsd:element name='street' type='xsd:string'/>		\\
	~~~~<xsd:element name='city'   type='xsd:string'/>		\\
	~~~~<xsd:element name='state'  type='xsd:string'/>		\\
	~~~~<xsd:element name='zip'    type='xsd:decimal'/>		\\
	~~</xsd:sequence>						\\
	~~<xsd:attribute name='country' type='xsd:NMTOKEN' fixed='US'/>	\\
	</xsd:complexType>
}

\pause

\begin{itemize}
\item	El tipo \texttt{USAddress} puede ser usado en otras partes del
	esquema.
	\pause

\item	Representa un elemento XML con la secuencia de subelementos
	\texttt{name}, \texttt{street}, \texttt{city}, \texttt{state} y
	\texttt{zip}.
	\pause

\item	Opcionalmente, puede tener un atributo \texttt{country}, pero sí o sí
	con el valor "\texttt{US}".
\end{itemize}

\end{frame}

\begin{frame}
\frametitle{Ejemplos - 2}
\footnotesize
\texttt{<xsd:complexType name='PurchaseOrderType'>			\\
	~~<xsd:sequence>						\\
	~~~~<xsd:choice>						\\
	~~~~~~<xsd:group   ref='shipAndBill'/>				\\
	~~~~~~<xsd:element name='singleUSAddress' type='USAddress'/>	\\
	~~~~</xsd:choice>						\\
	~~~~<xsd:element ref='comment' minOccurs='0'/>			\\
	~~~~<xsd:element name='items'  type='Items'/>			\\
	~~</xsd:sequence>						\\
	~~<xsd:attribute name='orderDate' type='xsd:date'/>		\\
	</xsd:complexType>						\\
	~								\\
	<xsd:group id='shipAndBill'>					\\
	~~<xsd:sequence>						\\
	~~~~<xsd:element name='shipTo' type='USAddress'/>		\\
	~~~~<xsd:element name='billTo' type='USAddress'/>		\\
	~~</xsd:sequence>						\\
	</xsd:group>
}
\end{frame}

\begin{frame}
\frametitle{Además de validar...}
\begin{itemize}

\item	Existe una herramienta que dado un un esquema XSD puede
	generar una clase C++ que lo representa, y un parser para
	documentos (XSD-CXX de CodeSynthesis).
	\pause

\item	\texttt{<contact>			\\
		~~<name>John Doe</name>		\\
		~~<email>j@doe.com</email>	\\
		~~<phone>555 12345</phone>	\\
		</contact>
	}
	\pause

\item	\texttt{auto\_ptr<Contact> c = contact(\dquote c.xml\dquote);	\\
		cout << c->name() << \dquote , \dquote 			\\
		~~~~~<< c->email() << \dquote , \dquote 		\\
		~~~~~<< c->phone() << endl;
}

\end{itemize}
\end{frame}
