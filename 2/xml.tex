\section{XML}

\begin{frame}
\frametitle{Introducción}
\begin{itemize}

\item	XML es un lenguaje de ``markup'', o sea, permite estructurar documentos
	y agregar metadatos de manera que sea sintáctimente distinguible del
	contenido.
	\pause

\item	Fue diseñado por el W3C con los objetivos de:
\begin{itemize}
	\item	Que sea fácilmente usable a través de toda la internet
	\item	Que pueda soportar distintos tipos de uso y aplicaciones
	\item	Que sea fácil escribir programas que usen documentos XML
	\item	Que sea legible por un humano
	\item	etc...
\end{itemize}
\pause

\item	En otras palabras, que sea {\it el} lenguaje estándar para intercambio
	de datos semiestructurados.
\end{itemize}
\end{frame}

\begin{frame}
\frametitle{Estructura}
\begin{itemize}

\item	Es conceptualmente un árbol, con una única raíz.
\pause

\item	El árbol no tiene una estructura predefinida (XML es de alguna manera
	un metalenguaje) y es lo más genérico posible. Cada nodo (usualmente
	llamado elemento) tiene:
	\pause
\begin{itemize}
	\item	Un nombre, que usualmente representa el ``tipo''
		del elemento.
		\pause
	\item	Opcionalmente, una lista de atributos.
		\pause
	\item	Su contenido, que puede estar compuesto por
		mas elementos.
\end{itemize}

\end{itemize}
\end{frame}

\begin{frame}
\frametitle{Ejemplos - 1}
\footnotesize
\texttt{<note>						\\
	~~<to>Tove</to>					\\
	~~<from>Jani</from>				\\
	~~<heading>Reminder</heading>			\\
	~~<body>Don't forget me this weekend!</body>	\\
	</note>
}
\end{frame}

\begin{frame}
\frametitle{Ejemplos - 2}
\footnotesize
\texttt{<CATALOG>				\\
	~~<CD>					\\
	~~~~<TITLE>Empire Burlesque</TITLE>	\\
	~~~~<ARTIST>Bob Dylan</ARTIST>		\\
	~~~~<COUNTRY>USA</COUNTRY>		\\
	~~~~<COMPANY>Columbia</COMPANY>		\\
	~~~~<PRICE>10.90</PRICE>		\\
	~~~~<YEAR>1985</YEAR>			\\
	~~</CD>					\\
	~~<CD>					\\
	~~~~<TITLE>Hide your heart</TITLE>	\\
	~~~~<ARTIST>Bonnie Tyler</ARTIST>	\\
	~~~~<COUNTRY>UK</COUNTRY>		\\
	~~~~<COMPANY>CBS Records</COMPANY>	\\
	~~~~<PRICE>9.90</PRICE>			\\
	~~~~<YEAR>1988</YEAR>			\\
	~~</CD>					\\
	</CATALOG>
}
\end{frame}

\begin{frame}
\frametitle{Ejemplos - 3}
\footnotesize
\texttt{<CATALOG>				\\
	~~<CD>					\\
	~~~~<TITLE>Empire Burlesque</TITLE>	\\
	~~~~<ARTIST>Bob Dylan</ARTIST>		\\
	~~~~<COUNTRY>USA</COUNTRY>		\\
	~					\\
	~~~~<PRICE>10.90</PRICE>		\\
	~					\\
	~~</CD>					\\
	~~<CD>					\\
	~~~~<TITLE>Hide your heart</TITLE>	\\
	~~~~<ARTIST>Bonnie Tyler</ARTIST>	\\
	~~~~<COUNTRY>UK</COUNTRY>		\\
	~~~~<COMPANY>CBS Records</COMPANY>	\\
	~					\\
	~~~~<YEAR>1988</YEAR>			\\
	~~</CD>					\\
	</CATALOG>
}
\end{frame}

\begin{frame}
\frametitle{Ejemplos - 4}
\footnotesize
\texttt{<td style=\dquote line-height:1.35em;\dquote>		\\
	~~<a href=\dquote /wiki/Markup\_language\dquote ~
		title=\dquote Markup language\dquote>		\\
	~~~~Markup language					\\
	~~</a>							\\
	</td>
}
\end{frame}

\begin{frame}
\frametitle{Parseo}
\begin{itemize}
	\item	Existen muchisímas librerías que parsean XML, por lo cual
		adaptar una aplicación existente a usar XML suele ser fácil
		(nunca tenemos que preocuparnos por parsearlo).
		\pause

	\item	Interpretar el XML es trabajo de la aplicación en cuestión
		(ya que la definición de XML no acarrea significado).
		\pause

	\item	Existen lenguajes de consulta que funcionan sobre XML
		(XQuery, XQL)
\end{itemize}
\end{frame}
