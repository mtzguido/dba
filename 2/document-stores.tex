\section{Document stores}

\begin{frame}
\frametitle{Mas allá de un KV-store}
\begin{itemize}

\item	Si bien los KV-stores tienen excelentes propiedades de performance
	y disponibilidad (pueden distribuirse automáticamente), a veces
	queremos más funcionalidad del DBMS. El hecho de que los valores
	sean opacos es justamente el problema.
	\pause

\item	Queremos que el motor sea consciente de la estructura interna
	del dato, y poder usarla para hacer consultas.
	\pause

\item	Las DBs de documentos no imponen un esquema sobre los datos de
	cada colección, permitiendo flexibilidad en los datos.
	\pause

\item	Esto tiene la ventaja sobre un RDBMS de poder hacer un despliegue
	rápido sin downtime. Esto es especialmente útil en los sistemas
	web de mayor tamaño, en donde hay cambio constante.
\end{itemize}
\end{frame}

\begin{frame}
\frametitle{Caracterización}
\begin{itemize}
\item	El concepto central de una Document Store es el ``Documento''.
	El DBMS guarda y devuelve documentos que pueden estar representados
	en XML, JSON, BSON o algún otro lenguaje para datos semi-estructurados.
	\pause

\item	Algunas también permiten replicación y sharding automático, por lo
	cual es bastante simple correrlas en un cluster.
	\pause

\item	Permiten agregar condiciones sobre los valores (o presencia) de los
	campos a las consultas, y proyectar la información.
	\pause

\item	También, permiten la creación de índices sobre los datos, lo cual
	puede permitir enormes ganancias en performance.
\end{itemize}
\end{frame}
