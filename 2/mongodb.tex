\section{MongoDB}

\begin{frame}
\frametitle{Introducción}
\begin{itemize}

\item	MongoDB es una DB orientada a documentos multi-plataforma
	considerada la más usada entre las NoSQL.
	\pause

\item	Usa documentos en formato JSON (internamente BSON) y no necesita
	ninguna declaración de esquema.
	\pause

\item	Los documentos se agregan a colecciones y automáticamente se les
	da un \texttt{\_id} único.
\end{itemize}
\end{frame}

\begin{frame}
\frametitle{Uso básico}
\begin{itemize}

\item	Agregar un documento a una colección: \\
	\texttt{\footnotesize
		> db.coll.insert(\{city: 'Rosario', country: 'Argentina'\})
	}
	\pause

\item	Recuperar toda la colección: \\
	\texttt{\footnotesize
		> db.coll.find()
	}
	\pause

\item	Recuperar sólo documentos con \texttt{country} igual a
	\texttt{'Argentina'} \\
	\texttt{\footnotesize
		> db.coll.find(\{country: 'Argentina'\}) \\
		\{ '\_id' : ObjectId('55469f06291c11c92d62d784'), 'city' : 'Rosario', 'country' : 'Argentina' \}
	}
	\pause

\item	Si queremos sólo el nombre de la ciudad: \\
	\texttt{\footnotesize
		> db.coll.find(\{country: 'Argentina'\}, \{city: 1\}) \\
		\{ 'city' : 'Rosario' \}
	}

\end{itemize}
\end{frame}
