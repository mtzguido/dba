\section{Bases de datos orientadas a Grafos (BDOG o Graph Databases)}

\begin{frame}
\frametitle{Introducción}
\begin{itemize}

\item	Algunas aplicaciones requieren guardar grafos gigantescos de
	forma segura y manipularlos eficientemente. \pause
	De hecho, podemos pensar que todo tipo de datos es realmente
	un grafo. Los modelos relacionales son entonces un caso
	particular de grafo.
	\pause

\item	Como siempre, se puede usar un motor relacional y manejar
	la lógica a nivel de aplicación. Se podrían incluso usar índices
	para acelerar los joins (que representarían la búsqueda de vecinos)
	a costo de updates mas lentos y espacio en disco.
	\pause

\item	Una BDOG se define como una base de datos que provee ``adjacencia
	sin uso de índices'', o sea, que maneja los grafos de una manera
	\textit{natural} y \textit{eficiente}.
	\pause

\item	De esta manera, se prestan a modelar datos heterogéneos y
	conectados.
\end{itemize}
\end{frame}

\begin{frame}
\frametitle{Un poco de autobombo}
\begin{itemize}
\item	We live in a connected world. There are no isolated pieces of
	information, but rich, connected domains all around us. Only a database
	that embraces relationships as a core aspect of its data model is able
	to store, process, and query connections efficiently. \textbf{While other
	databases compute relationships expensively at query time, a graph
	database stores connections as first class citizens, readily available
	for any “join-like” navigation operation.} Accessing those already
	persistent connections is an efficient, constant-time operation and
	allows you to quickly traverse millions of connections per second per
	core.
\end{itemize}
\end{frame}

\begin{frame}
\center
\includegraphics[width=0.9\linewidth,height=0.9\textheight,keepaspectratio]{bdog-1}
\end{frame}
\begin{frame}
\center
\includegraphics[width=0.9\linewidth,height=0.9\textheight,keepaspectratio]{bdog-2}
\end{frame}

\begin{frame}
\frametitle{Implementando un KV store}
 \center
 \includegraphics[width=0.9\linewidth,height=0.9\textheight,keepaspectratio]{bdog-3}
\end{frame}

\begin{frame}
\frametitle{Implementando un KV store}
 \center
 \includegraphics[width=0.9\linewidth,height=0.9\textheight,keepaspectratio]{bdog-4}
\end{frame}

\begin{frame}
\frametitle{Características}
\begin{itemize}
\item	En una BDOG, nunca puede haber una relación sin ambos nodos
	que conecta (\textit{No broken links}).
	\pause

\item	Pueden tener consistencia ACID (Neo4j la tiene).
	\pause

\item	Al tener punteros directos a otros nodos, no se requiere el uso
	de joins que pueden ser prohibitivamente costosos.
\end{itemize}
\end{frame}

\section{Neo4j}

\begin{frame}
\frametitle{Neo4j}
\begin{itemize}
\item	Neo4j es un GDBMS open-source, con una versión gratuita y una
	paga, que implementa este modelo. Da garantías ACID y escala
	horizontalmente \textit{en lectura} mediante clusters master-slave.
	\pause

\item	Usada por muchísimas empresas y organizaciones (eBay, Walmart,
	Cisco, y más).
	\pause

\item	Permite importar datos de motores relacionales con cierta
	facilidad.
	\pause

\item	Usa un lenguaje de consulta llamado \textit{Cypher} que permite
	hacer todas las operaciones que esperamos.
\end{itemize}
\end{frame}

\begin{frame}
\frametitle{Ejemplo rápido de Cypher}
\footnotesize

\texttt{SELECT name FROM Person					\\
	LEFT JOIN Person\_Department				\\
	~~ON Person.Id = Person\_Department.PersonId		\\
	LEFT JOIN Department					\\
	~~ON Department.Id = Person\_Department.DepartmentId	\\
	WHERE Department.name = 'IT Department'			\\
}
\vspace{5mm}
\pause
\texttt{MATCH (p:Person)<-[:EMPLOYEE]-(d:Department)		\\
	WHERE d.name = 'IT Department'				\\
	RETURN p.name
}

\end{frame}

\section{Demo de Neo4j}
