\section{Modelos de consistencia}

\begin{frame}
\frametitle{Introducción}
\begin{itemize}
\item	Debido al teorema CAP no podemos tener un sistema
	perfectamente consistente con tolerancia a fallas
	y disponibilidad. Sin embargo, podemos relajar algunas
	de estas propiedades para obtener modelos interesantes.
	\pause

\item	Estos modelos deben proporcionar alguna solución a los
	conflictos write-write, read-write y de réplica.
	\pause

\item	Muchas DBs implementan estos modelos relajados de consistencia,
	incluso con valores configurables para poder ajustarlas a nuestras
	necesidades.
\end{itemize}
\end{frame}

\begin{frame}
\frametitle{Quórums - WW}
\begin{itemize}
\item	Para evitar conflictos write-write, podemos hacer que los
	distintos nodos se sincronicen antes de efectuar la escritura.
	La cantidad de nodos necesaria se conoce como Quórum de escritura
	(\textbf{W}).
	\pause

\item	Si nuestro clúster tiene 5 nodos, basta con que 3 ``aprueben''
	la escritura para poder hacerla sin problemas. No puede haber otra
	escritura ``aprobada'' al mismo tiempo.
	\pause

\item	De hecho, sólo hace falta tener mayoría sobre los nodos que
	guardan el mismo registro. Si cada registro está triplicado
	en un sistema de mil nodos, sólo necesitamos 2 \textit{acks}.
\end{itemize}
\end{frame}

\begin{frame}
\frametitle{Quórums - Replicación}
\begin{itemize}
\item	Siguiendo esa idea, para evitar conflictos de replicación
	podemos pedir un Quórum de lectura (\textbf{R}).
	\pause

\item	En este caso, nos aseguramos que \textbf{R} nodos tengan
	sus datos consistentes antes de tomar la lectura como válida.
\end{itemize}
\end{frame}

\begin{frame}
\frametitle{Combinando}
\begin{itemize}
\item	Si llamamos \textbf{N} al factor de replicación del sistema,
	tenemos que:

	El sistema es consistente $ \iff R + W > N$
	\pause

\item	Esto abre a la puerta a debate sobre los valores óptimos
	para cada variable (y su impacto en latencia y tolerancia
	a particiones), para cada aplicación.
\end{itemize}
\end{frame}
