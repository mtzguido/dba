
\subsection{Motivación}

\begin{frame}
	\frametitle{Motivación}
	Siendo capaces de almacenar la ubicación de objetos y sus estados a lo largo del tiempo, naturalmente se espera poder combinar ambas posibilidades. \pause \\
	Esto resulta en un modelo de bases de datos espacio-temporales, en donde la geometría de los datos involucrados también queda sujeta a los tiempos de validez y transacción propios de las DBs temporales. \pause \\
Algunos ejemplos que justifican este modelo son: \pause
	\begin{itemize}
	\item	Seguimiento de objetos móviles (por ejemplo, satélites). \pause
	\item	Registro de movimiento de placas tectónicas \pause
	\item	Cruza de datos entre vuelos y pronósticos temporales
	\end{itemize}
\end{frame}

\begin{frame}
	\frametitle{Definición de DB espacio-temporal}
	Una base de datos espacio-temporal es aquella que provee soporte interno tanto para información temporal como para información geográfica. \pause \\
	Es, de los tres tipos de bases de datos expuestos, el menos desarrollado. No existe en el mercado ningún producto de RDBMS con extensiones espaciotemporales, y ni siquiera se ha desarrollado aún un estándar para su modelado y consulta.
\end{frame}

\begin{frame}
	\frametitle{STQL: un lenguaje de consultas espacio-temporales}
	Aunque no existe un estándar, se han propuesto lenguajes de consulta y modelado de datos espaciotemporales que abarcan los aspectos más importantes de este enfoque. \pause \\
	Uno de ellos, publicado por Martin Erwig y Markus Schneider en 2002, es \textbf{STQL: a Spatio-Temporal Query Language}. \pause \\
	En este lenguaje, un tipo de datos abstracto llamado \texttt{objeto móvil} representa a los objetos geométricos cuya posición o forma puede variar en el tiempo. \\
	Además, los investigadores proponen un método de extrapolar (``levantar'') los objetos espaciales agregándolos al dominio temporal, obteniendo objetos móviles y regiones mutantes. \pause \\
	Luego, se redefinen los operadores espaciales considerando su semántica original, pero agregando su resultado según la consulta. Por ejemplo, puede consultarse si ``alguna vez'' una línea intersectó una región o si ``siempre'' un punto estuvo dentro de un polígono.
\end{frame}

\begin{frame}
	\frametitle{Ejemplos de consultas espaciotemporales}
	Consideremos un escenario relacionado al control de fuego forestal. Trabajaremos con una base de datos conformada por las siguientes tablas: 
	\begin{itemize}
	\item \texttt{
		TABLE forest(forestname string, Territory Region);
	}

	\item \texttt{
		TABLE forest\_fire(firename string, Extent Region);
	}

	\item \texttt{
		TABLE fire\_fighter(figtername string, Location Point);
	}
	\end{itemize}

\end{frame}

\begin{frame}
	\frametitle{Ejemplos de consultas espaciotemporales: número 1}
	Si queremos conocer el área total destruida por el fuego ``Gran Fuego'', podemos ejecutar la siguiente consulta: \pause 
	\begin{itemize}
	\item \texttt{
		SELECT sum(size) FROM 
			(SELECT size AS area( 
			traversed(Intersection(Territory, Exent))) 
			FROM forest\_fire, forest 
			WHERE firename = ``Gran Fuego'' AND 
			ever(Intersects(Territory, Extent))
	}
	\end{itemize}
	La función \texttt{traversed}, retorna la proyección espacial de las sucesivas intersecciones entre el territorio del bosque y la región cubierta por ``Gran Fuego''. Al resultado, se le calcula el área.
\end{frame}

\begin{frame}
	\frametitle{Ejemplos de consultas espaciotemporales: número 2}
	Supongamos que queremos conocer cuánto tiempo el Teniente Miller estuvo dedicado a sofocar el incendio provocado por ``Gra Fuego''; y qué distancia recorrió en su misión. Entonces, ejecutamos la siguiente consulta: \pause 
	\begin{itemize}
	\item \texttt{
		SELECT time AS duration(dom( 
					Intersection(Location, GranFuego))), 
				distance AS length(trajectory( 
					Intersection(Location, GranFuego))) 
			FROM fire\_fighter 
			WHERE fightername = ``Tt. Miller''
	}
	\end{itemize}
	, asumiendo que GranFuego haya sido definido con anterioridad.
	El operador \texttt{dom} obtiene los intervalos en los que Location (un punto representando la posición del teniente) intersectó con GranFuego. Luego, \texttt{duration} suma estos intervalos. \\
	De manera similar, la función \texttt{trajectory} computa la proyección espacial del movimiento del teniente dentro de la región de GranFuego sobre el plano euclideano. Posteriormente, \texttt{length} calcula su longitud total.
\end{frame}
