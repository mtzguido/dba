
\subsection{Motivación}

\begin{frame}
\frametitle{Motivación}
\begin{itemize}
	\item	Siendo capaces de almacenar la ubicación de objetos y
	sus estados a lo largo del tiempo, naturalmente se espera poder
	combinar ambas posibilidades. \pause

	\item	Esto resulta en un modelo de bases de datos
	espacio-temporales, en donde la geometría de los datos involucrados
	también queda sujeta a los tiempos de validez y transacción propios
	de las DBs temporales. \pause

	\item	Algunos ejemplos que justifican este modelo son: \pause
	\begin{itemize}
	\item	Seguimiento de objetos móviles (por ejemplo, satélites). \pause
	\item	Registro de movimiento de placas tectónicas \pause
	\item	Cruza de datos entre vuelos y pronósticos temporales
	\end{itemize}
\end{itemize}
\end{frame}

\begin{frame}
\frametitle{Definición de DB espacio-temporal}
\begin{itemize}
	\item	Una base de datos espacio-temporal es aquella que provee
	soporte interno tanto para información temporal como para información
	geográfica. \pause \\

	\item	Es, de los tres tipos de bases de datos expuestos, el menos
	desarrollado. No existe en el mercado ningún producto de RDBMS con
	extensiones espaciotemporales, y ni siquiera se ha desarrollado aún un
	estándar para su modelado y consulta.
\end{itemize}
\end{frame}

\begin{frame}
\frametitle{STQL: un lenguaje de consultas espacio-temporales}
\begin{itemize}

	\item	Aunque no existe un estándar, se han propuesto lenguajes de
	consulta y modelado de datos espaciotemporales que abarcan los aspectos
	más importantes de este enfoque. \pause

	\item	Uno de ellos, publicado por Martin Erwig y Markus Schneider en 2002,
	es \textbf{STQL: a Spatio-Temporal Query Language}. \pause

	\item	En este lenguaje, un tipo de datos abstracto llamado
	\texttt{objeto móvil} representa a los objetos geométricos cuya posición
	o forma puede variar en el tiempo. \pause

	\item	Además, los investigadores proponen un método de extrapolar
	(``levantar'') los objetos espaciales agregándolos al dominio
	temporal, obteniendo objetos móviles y regiones mutantes. \pause

	\item	Luego, se redefinen los operadores espaciales considerando su
	semántica original, pero agregando su resultado según la consulta.
	Por ejemplo, puede consultarse si ``alguna vez'' una línea intersectó
	una región o si ``siempre'' un punto estuvo dentro de un polígono.
\end{itemize}
\end{frame}

\begin{frame}
\frametitle{Ejemplos de consultas espaciotemporales: número 1}
\begin{itemize}
	\item	Un ejemplo simple de información sobre vuelos de aerolíneas
		y clima.
	\pause

	\texttt{\small{
	\item	flights(id:string, Route:Point)		\\
		weather(kind:string, Extent:Region)
	}}
	\pause

	\item	¿Donde estuvo el vuelo UA207 a las 8AM?	\\
		\texttt{\small{SELECT Route(8:00) FROM flights	\\
			WHERE id = 'UA207'
		}}
	\pause

	\item	¿Cual fue el movimiento desde las 7AM a las 9AM? \\
		\texttt{\small{SELECT Route(7:00..9:00) FROM flights	\\
			WHERE id = 'UA207'
		}}
	\pause

	\item	Proyectar la ruta \\
		\texttt{\small{SELECT trajectory(Route(7:00..9:00)) FROM flights	\\
			WHERE id = 'UA207'
		}}
\end{itemize}
\end{frame}

\begin{frame}
\frametitle{Ejemplos de consultas espaciotemporales}
\begin{itemize}
	\item	Ahora consideremos un escenario relacionado al control de fuego
		forestal. Trabajaremos con una base de datos conformada por las
		siguientes tablas:

	\item \texttt{TABLE forest(forestname string, Territory Region);}

	\item \texttt{TABLE forest\_fire(firename string, Extent Region);}

	\item \texttt{TABLE fire\_fighter(fightername string, Location Point);}
\end{itemize}
\end{frame}

\begin{frame}
\frametitle{Ejemplos de consultas espaciotemporales: número 2}
\begin{itemize}
	\item	Si queremos conocer el área total destruida por el fuego
		``Gran Fuego'', podemos ejecutar la siguiente consulta: \pause

	\texttt{\footnotesize{
	\item	SELECT sum(size) FROM						\\
		~(SELECT~size							\\
		~~~~~~~~~AS area(traversed(Intersection(Territory, Extent)))	\\
		~~FROM~~~forest\_fire, forest					\\
		~~WHERE~~firename = 'Gran Fuego' AND				\\
		~~~~~~~~~ever(Intersects(Territory, Extent))
	}}
	\pause

	\item	La función \texttt{traversed}, retorna la proyección espacial
		de las sucesivas intersecciones entre el territorio del bosque
		y la región cubierta por ``Gran Fuego''. Al resultado, se le
		calcula el área.
\end{itemize}
\end{frame}

\begin{frame}
\frametitle{Ejemplos de consultas espaciotemporales: número 3}
\begin{itemize}
	\item	Supongamos que queremos conocer cuánto tiempo el Teniente Miller
		estuvo dedicado a sofocar el incendio provocado por ``Gran Fuego'';
		y qué distancia recorrió en su misión. Entonces, ejecutamos la
		siguiente consulta: \pause

	\texttt{\footnotesize{
	\item	SELECT time AS duration(dom(Intersection(Location, GranFuego))),
		distance AS length(trajectory(Intersection(Location, GranFuego)))
		FROM fire\_fighter WHERE fightername = 'Tt. Miller'}}
		(asumiendo que GranFuego haya sido definido con anterioridad)
	\pause

	\item	El operador \texttt{dom} obtiene los intervalos en los que Location
		(un punto representando la posición del teniente) intersectó con
		GranFuego. Luego, \texttt{duration} suma estos intervalos. \\
	\pause

	\item	De manera similar, la función \texttt{trajectory} computa la
		proyección espacial del movimiento del teniente dentro de la región
		de GranFuego sobre el plano euclideano. Posteriormente, \texttt{length}
		calcula su longitud total.
\end{itemize}
\end{frame}
