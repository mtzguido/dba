
\subsection{Motivación}

\begin{frame}
	\frametitle{Motivación}
	Siendo capaces de almacenar la ubicación de objetos y sus estados a lo largo del tiempo, naturalmente se espera poder combinar ambas posibilidades. \pause \\
	Esto resulta en un modelo de bases de datos espacio-temporales, en donde la geometría de los datos involucrados también queda sujeta a los tiempos de validez y transacción propios de las DBs temporales. \pause \\
Algunos ejemplos que justifican este modelo son: \pause
	\begin{itemize}
	\item	Seguimiento de objetos móviles (por ejemplo, satélites). \pause
	\item	Registro de movimiento de placas tectónicas \pause
	\item	Cruza de datos entre vuelos y pronósticos temporales
	\end{itemize}
\end{frame}

\begin{frame}
	\frametitle{Definición de DB espacio-temporal}
	Una base de datos espacio-temporal es aquella que provee soporte interno tanto para información temporal como para información geográfica. \pause
	Es, de los tres tipos de bases de datos expuestos, el menos desarrollado. No existe en el mercado ningún producto de RDBMS con extensiones espaciotemporales, y ni siquiera hay un estándar para su modelado y consulta. \pause
\end{frame}

\begin{frame}
	\frametitle{STQL: un lenguaje de consultas espacio-temporales}
	Aunque no existe un estándar, se han propuesto lenguajes de consulta y modelado de datos espaciotemporales que abarcan los aspectos más importantes de este enfoque. \pause \\
	Uno de ellos, publicado por Martin Erwig y Markus Schneider en 2002, es \textbf{STQL: a Spatio-Temporal Query Language}. \pause \\
	En este lenguaje, un tipo de datos abstracto llamado \texttt{objeto móvil} representa a los objetos geométricos cuya posición o forma puede variar en el tiempo.

\end{frame}
